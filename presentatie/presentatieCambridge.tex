\documentclass[mathserif,professionalfont]{beamer}
\usepackage{pxfonts} 
%\usepackage{eulervm}
\usepackage[english]{babel}
\usepackage{calc}
\usepackage[absolute,overlay]{textpos}
\mode<presentation>{\usetheme{tud}}
%\usepackage{subcaption}
\usepackage{tcolorbox}
\usepackage{tikz}
\usetikzlibrary{positioning,arrows}
\usepackage[utf8x]{inputenc}
\usepackage{scalefnt}
\usetikzlibrary{decorations.markings}
\usetikzlibrary{shapes,snakes}
\usetikzlibrary{shapes.geometric}
\usetikzlibrary{fit}					
\usetikzlibrary{backgrounds}
\usetikzlibrary{positioning}
\usetikzlibrary{arrows}
\usepackage{pgffor}
\usepackage{amsmath,amssymb,mathrsfs}
\usepackage{amsthm}
\usepackage{color}
\usepackage{wasysym}
\usepackage{pdfpages}
\usepackage{paralist}
\usepackage{framed}
\usepackage{fancybox} %kaders
\usepackage{pifont}
\usepackage{pgfplots}
\usepackage{listofsymbols}
\usepackage{multirow}
%\usepackage[pdftex]{hyperref}
%\usepackage{enumitem}
\tikzstyle{every picture}+=[remember picture]
\tikzstyle{na} = [baseline=-.5ex]


\title[Accuracy of original MPM]{Accuracy of original MPM}
%\institute[]{}
\author[]{Lisa Wobbes, Roel Tielen}
\date[\today]{\today}

\definecolor{blueM}{rgb}{0.4, 0.6, 0.8}

\begin{document}
%titel----------------------------------------------------------------------------------------------------------------------------------------------------------------
{
%\usebackgroundtemplate{\includegraphics[width=\paperwidth,height=\paperheight]{logos}}%
%\setbeamertemplate{footline}{\usebeamertemplate*{minimal footline}}
\frame{\titlepage}
}

%outline------------------------------------------------------------------------------------------------------------------------------------------------------------
\begin{frame}{Outline}
\begin{itemize}
\item Numerical accuracy
\item Benchmarks
	\begin{itemize}
	\item Vibrating linear-elastic bar 
	\item Vibrating hyper-elastic bar
	\item Oedometer
	\end{itemize}
\item
\item
\item
\end{itemize}
\end{frame}

%------------------------------------------------------------------------------------------------------------------
\begin{frame}{Numerical accuracy}
\begin{tcolorbox}[colback=red!5,colframe=red!50!black,title=Numerical Approximation]
$u_{ex} = u_{num} + \mathcal{O}(\Delta x^n) + \mathcal{O}(\Delta t)$ %with $n \leq 2$
\end{tcolorbox}
\begin{tcolorbox}[colback=red!5,colframe=red!50!black,title=RMS Error]
$Error_{RMS} = \sqrt{\frac{1}{n_p} \left(\sum_{p=1}^{n_p}u_{num}(x_p,t) - u_{ex}(x_p,t)\right)^2}$
\end{tcolorbox}
\begin{tcolorbox}[colback=blue!5,colframe=blue!40!black,title=Accuracy in displacement]
For $\Delta t \to 0$, the order of accuracy is equal to $n$, i.e. the reduction of $\Delta x$ by a factor of 2 decreases the RMS error by $2^n$.
\end{tcolorbox}
\end{frame}

%------------------------------------------------------------------------------------------------------------------
\begin{frame}{Vibrating linear-elastic bar}
  \begin{minipage}{\linewidth}
      \centering
      \begin{minipage}{0.45\linewidth}
              \definecolor{darkred}{cmyk}{0,0.9,0.9,0.2}
\definecolor{darkblue}{cmyk}{0.9,0.5,0.1,0.2}
\tikzset{
mystyle1/.style={
  rectangle,
  inner sep=0.05pt,
  text width=1mm,
  fill=black
  }
}

\tikzset{
mystyle2/.style={
  circle,
  inner sep=0.1pt,
  text width=2mm,
  fill=red!80
  }
}
\tikzstyle{myarrows}=[line width=0.4mm,draw=darkred,postaction={draw, line width=0.4mm, shorten >=3mm, -}]
\begin{figure}[H]
\centering
\begin{tikzpicture}[scale=0.48]
 %structure
 \draw[fill=black!25] (0.5,2) rectangle (6.5,2.55);
 \draw[fill=darkblue] (-0.3,3.5) rectangle (0.5,1);
%initial velocity
\foreach \i in {1,...,5}
{
        \pgfmathsetmacro{\z}{0.5 + (6/5)*\i};
        \pgfmathsetmacro{\y}{2.275};
        \pgfmathsetmacro{\l}{1.5*sin((175*\z)/12)};
        \draw[myarrows] (\z,\y) -- (\z,\y+\l);
	\draw[->, ultra thick,darkred] (\z,\y+1.01*\l) -- (\z,\y+1.05*\l);
}
\draw (7.2,3.2) node {$v_0$};
%coordinate
\draw[->, thick] (-0.3,0.2)--(7.8,0.2);
\draw (7.5, -0.4) node {$x$};
 %coordinates
 \draw[thick] (0.5, 0.05)--(0.5,0.35);
 \draw (0.5,-0.4) node {0};
 \draw[thick] (6.5, 0.05)--(6.5,0.35);
 \draw (6.5,-0.4) node {$L$};
\end{tikzpicture}
\end{figure}
      \end{minipage}
      \hspace{0.01\linewidth}
      \begin{minipage}{0.45\linewidth}
         \begin{align}\nonumber
&\frac{\partial^2 u}{\partial t^2} = \frac{E}{\rho}\frac{\partial^2 u}{\partial x^2} \\ \nonumber
&\mbox{Boundary conditions: } \\  \nonumber
& u(0,t) = 0\\ \nonumber
& \frac{\partial u}{\partial x}(L,t) = 0\\  \nonumber
&\mbox{Initial conditions:}\\  \nonumber
& u(x,0) = 0\\ \nonumber
& \frac{\partial u}{\partial t}(x,0) = v_0\sin \left( \frac{\pi x}{2L} \right)
\end{align} 
      \end{minipage}
  \end{minipage}
\end{frame}

%------------------------------------------------------------------------------------------------------------------
\begin{frame}{Vibrating hyper-elastic bar}
  \begin{minipage}{\linewidth}
      \centering
      \begin{minipage}{0.45\linewidth}
\definecolor{darkred}{cmyk}{0,0.9,0.9,0.2}
\definecolor{darkblue}{cmyk}{0.9,0.5,0.1,0.2}
\tikzset{
mystyle1/.style={
  rectangle,
  inner sep=0.05pt,
  text width=1mm,
  fill=black
  }
}

\tikzset{
mystyle2/.style={
  circle,
  inner sep=0.1pt,
  text width=2mm,
  fill=red!80
  }
}
\tikzstyle{myarrows}=[line width=0.4mm,draw=darkred,postaction={draw, line width=0.4mm, shorten >=3mm, -}]
\begin{figure}[h]
\centering
\begin{tikzpicture}[scale = 0.48]
%structure
 \draw[fill=black!25] (0.5,2) rectangle (6.5,2.55);
 \draw[fill=darkblue] (-0.3,3.5) rectangle (0.5,1);
%traction force	
\draw[->, ultra thick,darkred] (6.5,3.2) -- (8.2,3.2);
 \draw (7.15,3.9) node {$F_{trac}$};
%coordinate
\draw[->, thick] (-0.3,0.4)--(8.2,0.4);
\draw (7.9, -0.7) node {$x$};
 %coordinates
 \draw[thick] (0.5, 0.05)--(0.5,0.35);
 \draw (0.5,-0.4) node {0};
 \draw[thick] (6.5, 0.05)--(6.5,0.35);
 \draw (6.5,-0.4) node {$L$};
\end{tikzpicture}
\end{figure}
      \end{minipage}
      \hspace{0.01\linewidth}
      \begin{minipage}{0.45\linewidth}
         \begin{align}\nonumber
&\frac{\partial^2 u}{\partial t^2} = \frac{E}{\rho}\frac{\partial^2 u}{\partial x^2} \\ \nonumber
&\mbox{Boundary conditions: } \\  \nonumber
& u(0,t) = 0\\ \nonumber
& \frac{\partial u}{\partial x}(L,t) = \frac{\tau}{\rho} \sin \left(\frac{\pi t}{L}\right)\\ \nonumber
&\mbox{Initial conditions:}\\  \nonumber
& u(x,0) = 0\\ \nonumber
& \frac{\partial u}{\partial t}(x,0) = 0
\end{align} 
      \end{minipage}
  \end{minipage}
\end{frame}




%------------------------------------------------------------------------------------------------------------------

\begin{frame}{Oedometer}
  \begin{minipage}{\linewidth}
      \centering
      \begin{minipage}{0.45\linewidth}
\definecolor{soil}{cmyk}{0,0.2,0.6,0.2}
\definecolor{darkred}{cmyk}{0,0.9,0.9,0.2}
\tikzset{
mystyle1/.style={
  rectangle,
  inner sep=0.05pt,
  text width=1mm,
  fill=black
  }
}

\tikzset{
mystyle2/.style={
  circle,
  inner sep=0.1pt,
  text width=2mm,
  fill=red!80
  }
}
\tikzstyle{myarrows}=[line width=0.4mm,draw=darkred,postaction={draw, line width=0.4mm, shorten >=3mm, -}]
\begin{figure}[h]
\centering
\begin{tikzpicture}
 %structure
 \draw[fill=black] (0.4,2.2) rectangle (2.6,-1.1);
 \draw[fill=soil] (0.5,2) rectangle (2.5,-1);
 \draw[fill=black!25] (0.5,2) rectangle (2.5,2.35);
 \draw (1.5, 0.5) node {soil};

 %load
 \draw[myarrows] (0.7,3) -- (0.7,2.35);
 \draw[myarrows] (1.5,3) -- (1.5,2.35);
 \draw[myarrows] (2.3,3) -- (2.3,2.35);
\draw[->, ultra thick,darkred] (0.7,2.35) -- (0.7,2.34);
\draw[->, ultra thick,darkred] (1.5,2.35) -- (1.5,2.34);
\draw[->, ultra thick,darkred] (2.3,2.35) -- (2.3,2.34);
 \draw (1.55,3.3) node {$p_0$};
%coordinate axis
\draw[->, thick] (-1,-1.2)--(-1,3.3);
\draw (-1.3, 3.1) node {$y$};
 %coordinates
\draw[thick] (-1.1,2)--(-0.9,2); 
\draw (-1.3,2) node {$H$};
\draw[thick] (-1.1,-1)--(-0.9,-1);
\draw (-1.3,-1) node {0};
\end{tikzpicture}
\end{figure}
      \end{minipage}
      \hspace{0.01\linewidth}
      \begin{minipage}{0.45\linewidth}
         \begin{align}\nonumber
&\frac{\partial^2 u}{\partial t^2} = \frac{E}{\rho} \frac{\partial^2 u}{\partial y^2} -  g \\ \nonumber
&\mbox{Boundary conditions: } \\  \nonumber
& u(0,t) = 0\\ \nonumber
& \frac{\partial u}{\partial x}(L,t) = \frac{\tau}{\rho} \sin \left(\frac{\pi t}{L}\right)\\ \nonumber
&\mbox{Initial conditions:}\\  \nonumber
& u(y,0)=0,\\ \nonumber
& \frac{\partial u}{\partial t}(y,0)=0
\end{align} 
      \end{minipage}
  \end{minipage}
\end{frame}


%------------------------------------------------------------------------------------------------------------------

%-------------------------------------------------------------------------------------------------------------------
\begin{frame}{Grid-crossing}
\begin{center}
\begin{tikzpicture}[scale = 0.4]
\draw (0,0) coordinate(a_1) -- (10,0) coordinate(a_2);
\draw (a_1) -- (20,0) coordinate (a_3);

\draw (0,-1) coordinate (b_1);
\draw (20,-1) coordinate (b_3);
\draw (10,-1) coordinate (b_2);

\draw (b_1) node[below] {$x_{i-1}$};
\draw (b_3) node[below] {$x_{i+1}$} ;
\draw (b_2) node[below] {$x_{i}$};

\node[draw,circle,inner sep=3pt,fill] at (a_1) {};
\node[draw,circle,inner sep=3pt,fill] at (a_2) {};
\node[draw,circle,inner sep= 3pt,fill] at (a_3) {};

\node[draw,circle,inner sep=2.5pt,fill, color =orange] at (4,0) {};
\node[draw,circle,inner sep=2.5pt,fill, color = red] at (7.5,0) {};
\node[draw,circle,inner sep=2.5pt,fill, color =orange] at (14.5,0) {};
\node[draw,circle,inner sep=2.5pt,fill, color =orange] at (17,0) {};

 \draw[-to] (8,1) to[bend left] (12,1);

\end{tikzpicture}
\end{center}

\begin{center}
\begin{tikzpicture}[scale = 0.4]
\draw (0,0) coordinate(a_1) -- (10,0) coordinate(a_2);
\draw (a_1) -- (20,0) coordinate (a_3);

\draw (0,-1) coordinate (b_1);
\draw (20,-1) coordinate (b_3);
\draw (10,-1) coordinate (b_2);

\draw (b_1) node[below] {$x_{i-1}$};
\draw (b_3) node[below] {$x_{i+1}$} ;
\draw (b_2) node[below] {$x_{i}$};

\node[draw,circle,inner sep=3pt,fill] at (a_1) {};
\node[draw,circle,inner sep=3pt,fill] at (a_2) {};
\node[draw,circle,inner sep= 3pt,fill] at (a_3) {};

\node[draw,circle,inner sep=2.5pt,fill, color =orange] at (5,0) {};
\node[draw,circle,inner sep=2.5pt,fill, color = red] at (13,0) {};
\node[draw,circle,inner sep=2.5pt,fill, color =orange] at (15,0) {};
\node[draw,circle,inner sep=2.5pt,fill, color =orange] at (18.25,0) {};



\end{tikzpicture}
\end{center}
\end{frame}


%-------------------------------------------------------------------------------------------------------------------
\begin{frame}{Grid-crossing: properties of shape functions}
\begin{center}
\begin{tikzpicture}[scale=1.5]
    % Draw axes
    \draw [-,thick] (0,1.3) node (yaxis) [left] {$N_i$}
        |- (4,0) node (xaxis) [right] {};
    % Draw two intersecting lines
    \draw (0,0) coordinate (a_1) -- (2,1) coordinate (a_2);
    \draw (2,1) coordinate (b_1) -- (4,0) coordinate (b_2);
    \draw (2,0) coordinate (b_3);
    \draw (0,-0.2) node[below] {$x_{i-1}$};
    \draw (2,-0.2) node[below] {$x_{i}$};
    \draw (4,-0.2) node[below] {$x_{i+1}$};
    
 \shade[bottom color=gray!10, top color=black!80] (0,0) --(2,0) --(2,1);
 \shade[bottom color=gray!10, top color=black!80] (4,0) --(2,0) --(2,1);

\node[draw,circle,inner sep=3pt,fill] at (b_3) {};
\node[draw,circle,inner sep=3pt,fill] at (b_2) {};
\node[draw,circle,inner sep=3pt,fill] at (a_1) {};

\draw[dashed] (yaxis |- b_1) node[left] {$1$}
        -| (xaxis -| b_1) node[below] {};

\end{tikzpicture}
\end{center}

\begin{center}
\begin{tikzpicture}[scale=1.5]
 \shade[bottom color=gray!10, top color=black!80] (0,0)rectangle +(2,1);
 \shade[top color=gray!10, bottom color=black!80] (2,0) rectangle +(2,-1);
    % Draw axes
    \draw [-,thick] (0,-1.2) -- (0,1.2) node (yaxis) [left] {$\nabla N_i$}
        |- (4,0) node (xaxis) [right] {};

    % Draw two intersecting lines
    \draw (0,1) coordinate (a_1) -- (2,1) coordinate (a_2);
    \draw (2,-1) coordinate (b_1) -- (4,-1) coordinate (b_2);
    \draw (2,0) coordinate (b_3);

    \draw (-0.25,-0.2) node[below] {$x_{i-1}$};
    \draw (1.8,-0.2) node[below] {$x_{i}$};
    \draw (4.25,-0.2) node[below] {$x_{i+1}$};
    


\node[draw,circle,inner sep=3pt,fill] at (0,0) {};
\node[draw,circle,inner sep=3pt,fill] at (2,0) {};
\node[draw,circle,inner sep=3pt,fill] at (4,0) {};


\end{tikzpicture}
\end{center}


\end{frame}

%-------------------------------------------------------------------------------------------------------------------
\begin{frame}{Grid crossing: internal force}
 \begin{align}\nonumber
  & F^{int}_i \approx \sum_{p=1}^{n_i} \nabla N_i(\epsilon_p)\sigma_p \Omega_p + \sum_{p=1}^{n_{i+1}}\nabla N_i(\epsilon_p) \sigma_p \Omega_p\\ \nonumber
  & F^{int}_i \approx \sigma \Omega (n_i - n_{i+1})\\ \nonumber
  & \: \\ \nonumber
  & \begin{cases}F^{int}_i = 0, &\mbox{if } n_{i} = n_{i+1} \\
F^{int}_i \neq 0, & \mbox{otherwise} \end{cases}  
 \end{align}
\end{frame}







\end{document}
\grid
